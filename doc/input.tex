\documentclass[12pt]{article}
\usepackage{a4}
%-----------------------------------------------------------------------
% Define a bundle of useful commands.
%-----------------------------------------------------------------------
\newcommand*\key[1]{\mbox{\texttt{{#1}}}}  % keywords, labels
\newcommand*\subs[1]{\mbox{\textit{{#1}}}} % to be substituted by reader
\newcommand*\prg[1]{\mbox{\textbf{{#1}}}}  % programs, scripts, options
\newcommand*\file[1]{\mbox{\small \textsf{{#1}}}} % files, directories
\newcommand*\code[1]{\mbox{\texttt{{#1}}}} % routines, variables, code
\newcommand*\name[1]{\mbox{\textsc{{#1}}}} % persons, "commercial" progs
\newcommand*\myldots{\mbox{$\ldots$}}
\renewcommand*\mathbf[1]{\mbox{\boldmath$#1$\unboldmath}}
\renewcommand*\kappa{\varkappa}

\begin{document}

\section*{Input Keywords}

\noindent
The calculation to be performed is controlled via a number of keywords
and associated arguments.

\vspace{0.2cm}
\noindent
In general, any number of keywords may be present on a single line, in
free format.

\vspace{0.2cm}
\noindent
All blank lines are ignored. All text following a \code{\#} symbol is
ignored, that is, the \code{\#} symbol can be used to provide
comments.

\vspace{0.2cm}
\noindent
The input file must be terminated with the single keyword
\code{end-input}.

\subsection*{Selecting the level of perturbation theory}
\noindent
The level of perturbation theory is controlled by the \code{method} keyword.

\begin{table}[h]
\vspace*{1.7ex}
\begin{center}
\begin{tabular}{llp{2.5in}}
\code{method =} & \code{adc1}   & selects ADC(1) \\
                & \code{adc2-s} & selects ADC(2)-s \\
                & \code{adc2-x} & selects ADC(2)-x \\
\end{tabular}
\end{center}
\end{table}

\subsection*{Initial state symmetry}
\noindent
The symmetry species of the initial state is specified using the
keyword \code{istate\_symm}.

\begin{table}[h]
\vspace*{1.7ex}
\begin{center}
\begin{tabular}{llp{2.5in}}
\code{istate\_symm =} & \code{n}   & where \code{n} is the integer labeling the initial state symmetry \\
\end{tabular}
\end{center}
\end{table}

\subsection*{Final state symmetry}
\noindent
The symmetry of the final states is determined via the specification
of the symmetry species of the dipole operator using the keyword \code{dipole\_symm}.

\begin{table}[h]
\vspace*{1.7ex}
\begin{center}
\begin{tabular}{llp{2.5in}}
\code{dipole\_symm =} & \code{n}   & where \code{n} is the integer labeling the dipole operator symmetry \\
\end{tabular}
\end{center}
\end{table}

\subsection*{Initial state number}
\noindent
The number of the initial state, within the initial space symmetry
block, is specified by the keyword \code{initial\_state}.

\begin{table}[h]
\vspace*{1.7ex}
\begin{center}
\begin{tabular}{llp{2.5in}}
\code{initial\_state =} & \code{n}   & where \code{n} is state number of interest \\
\end{tabular}
\end{center}
\end{table}

\subsection*{Dipole operator}
\noindent
The component of the dipole operator is specified using the keyword \code{dipole\_component}.

\begin{table}[h]
\vspace*{1.7ex}
\begin{center}
\begin{tabular}{llp{2.5in}}
\code{dipole\_component =} & \code{c}   & where \code{c} is \code{x, y} or \code{z} \\
\end{tabular}
\end{center}
\end{table}

\subsection*{Controlling the block-Davidson procedure}
\noindent
The keywords controlling the block-Davidson calculation must appear in
a separate section between the keywords

\vspace{0.2cm}
\code{davidson\_section}

\vspace{0.2cm}
\noindent
and

\vspace{0.2cm}
\code{end-davidson\_section}

\vspace{0.2cm}
\noindent
Possible keywords are:

\begin{table}[h]
\vspace*{1.7ex}
\begin{center}
\begin{tabular}{llp{2.5in}}
\code{nstates =} & \code{n}   & where \code{n} is number of Davidson states to be calculated \\
\vspace{0.2cm}
\code{block\_size =} & \code{n} & where \code{n} is the block size to be used \\
\vspace{0.2cm}
\code{maxit =} & \code{n} & where \code{n} is the maximum number of iterations \\ 
\vspace{0.2cm}
\code{tol =} & \code{f} & where \code{f} is the tolerance for convergence \\
\vspace{0.2cm}
\code{guess =} & \code{adc1} & requests that the initial vectors are generated from the ADC(1) eigenvectors \\
\vspace{0.2cm}
\code{target =} & \code{f} & eigenpairs with energies closest to the target value \code{f} (in eV) will be found \\
\end{tabular}
\end{center}
\end{table}

\subsection*{Controlling the block-Lanczos procedure}
\noindent
The keywords controlling the block-Lanczos calculation must appear in
a separate section between the keywords

\vspace{0.2cm}
\code{lanczos\_section}

\vspace{0.2cm}
\noindent
and

\vspace{0.2cm}
\code{end-lanczos\_section}

\vspace{0.2cm}
\noindent
Possible keywords are:

\begin{table}[h]
\vspace*{1.7ex}
\begin{center}
\begin{tabular}{llp{2.5in}}
\code{ block\_size =} & \code{n} & where \code{n} is the block size to be used \\
\vspace{0.2cm}
\code{ iter =} & \code{n} & where \code{n} is number of iterations to be taken \\
\vspace{0.2cm}
\code{ mem =} & \code{M} & where M is the maximum amount of memory (in Mb) to be used (default 250 Mb) \\
\vspace{0.2cm}
$^{\dagger}$\code{guess =} & \code{adc1} & form the initial Lanczos vectors from the most important ADC(1) eigenvectors \\
               & \code{adc1\_mix} & form the initial Lanczos vectors from linear combinations of the most important ADC(1) eigenvectors and 2h2p intermediate states\\
               & \code{is\_mix} & form the initial Lanczos vectors from linear combinations of the most important 1h1p and 2h2p intermediate states \\
\vspace{0.2cm}
\code{ ortho =} & \code{pro} & use the partial reorthogonalisation algorithm \\
                & \code{mpro} & use the modified partial reorthogonalisation algorithm \\
\end{tabular}
\end{center}
\end{table}

\vspace{0.2cm}
\noindent
$^{\dagger}$ This is only relevant for ADC(2)-s calculations, and as
a default unit vectors corresponding to the most important 1h1p
intermediate states are taken as the initial vectors. For ADC(2)-x
calculations, the initial vectors are constrained to correspond to
unit vectors corresponding to the most important 1h1p and 2h2p
intermediate states.

\subsection*{Core-valence separation}
\noindent
The core-valence separation approximation is turned by by the single
keyword 

\vspace{0.2cm}
\code{cvs}

\subsection*{Calculation of ionization potentials}
\noindent
The calculation of ionization potentials via excitation into an
extremely diffuse `continuum' orbital is requested by the single
keyword

\vspace{0.2cm}
\code{fakeip}

\subsection*{Disabling of the calculation of Lanczos pseudo-spectra}
\noindent
The calculation of Lanczos pseudo-spectra can be disabled by the
single keyword 

\vspace{0.2cm}
\code{energy\_only}

\subsection*{Disabling the calculation of transition dipole moments to
the initial space states}
\noindent
The calculation of transition dipole moments between the ground state
and the initial space states may be disabled using the keyword

\vspace{0.2cm}
\code{no\_tdm}

\end{document}
